\documentclass{article}
\usepackage[utf8]{inputenc}
\usepackage[spanish]{babel}
\usepackage{geometry}
\geometry{a4paper, margin=1in}
\usepackage{hyperref}
\usepackage{enumitem}

	itle{Bitácora de Desarrollo: Proyecto de Monitoreo de Flujo de Cal}
\author{Gemini CLI}
\date{5 de Febrero de 2026, 21:02}

\begin{document}

\maketitle

\section*{1. Introducción al Proyecto}

El objetivo principal es desarrollar un software para monitorear el flujo de la cal dentro de una minera. Se cuenta con:
\begin{itemize}
    \item Una carpeta "Antiguo" con un trabajo previo abandonado (Fase 1 casi completa, Fase 2 con errores, a rehacer).
    \item Cuatro documentos PDF relevantes para el proyecto.
    \item Una carpeta "Informe" con una plantilla LaTeX para documentar el progreso.
    \item Una carpeta "cal\_monitoring\_backend" generada previamente, que será el punto de partida del nuevo software.
\end{itemize}

\section*{2. Revisión de Documentación Clave (PDFs)}

\subsection*{2.1. Análisis de ``40202412-La-Cal-Es-Un-Reactivo-Parte-I.pdf''}
Este documento es un texto técnico exhaustivo sobre la cal. Aportó una base teórica sólida, destacando:
\begin{itemize}
    \item La cal como reactivo fundamental en minería, especialmente en procesos hidrometalúrgicos.
    \item La importancia de la hidratación (apagado) y la formación de lechada de cal, con énfasis en el control de variables como temperatura, proporción agua/cal, agitación y calidad del agua.
    \item Los desafíos y variables críticas en el transporte hidráulico de la lechada de cal (granulometría, viscosidad, pH, incrustaciones, corrosión).
    \item La necesidad de instrumentación para monitorear y controlar estos procesos.
\end{itemize}
Este PDF subraya que el monitoreo del flujo de cal no es solo cantidad, sino también las propiedades de la lechada y las condiciones de su preparación y transporte.

\subsection*{2.2. Análisis de ``Especificación Instrumentación Y Software – Etapas 2 Y 3.pdf''}
Documento crucial que define el alcance y los requisitos del software. Se estructura en dos etapas:
\begin{itemize}
    \item 	extbf{Etapa 2 (Supervisión Integral y Análisis Operacional):} Se enfoca en la instrumentación existente. Incluye dashboards en tiempo real, registro histórico, alertas automáticas (nivel, flujo anómalo, equipos fuera de servicio) y análisis básico de causas. Menciona explícitamente la instrumentación actual (LT, FT, pHT, DT).
    \item 	extbf{Etapa 3 (Control Avanzado y Optimización del Sistema):} Requiere instrumentación adicional para control PID/cascada, ajuste automático de setpoints, detección predictiva de fallas y optimización.
\end{itemize}
Este documento es la hoja de ruta directa para la construcción del software, traduciendo los conceptos del primer PDF en requisitos concretos.

\subsection*{2.3. Análisis de ``Filosofía de control 4-V2-2270-IC-IPC-136001\_0\_PDF.pdf''}
Este documento detalla la filosofía de control para la planta de lechada de cal, proporcionando el "cómo" se controlan los procesos. Sus puntos clave son:
\begin{itemize}
    \item 	extbf{Arquitectura de Control:} Sistema de Control Distribuido (DCS), con instrumentación conectada vía FIELDBUS FOUNDATION y cableado directo.
    \item 	extbf{Modos de Operación:} Manual (Mantenimiento) y Automático (Operación).
    \item 	extbf{Alarmas e Interlocks:} Tablas extensas con equipos, tags, condiciones (ej. parada de emergencia, sobrecarga, niveles críticos, temperaturas, flujos) y las acciones/alarmas resultantes en el DCS.
    \item 	extbf{Secuencias Operacionales:} Descripciones detalladas de las secuencias de carga del silo, descarga a pre-mezclador, preparación de lechada (con control de temperatura por PID y alarmas), y sistema de dilución.
\end{itemize}
Este PDF visualiza la implementación práctica de los conceptos de los PDFs anteriores, proporcionando tags específicos y lógicas de operación.

\subsection*{2.4. Análisis de ``flujo\_mineria.pdf''}
Es un Diagrama de Flujo de Proceso (PFD) para la planta de lechada de cal. Complementa los documentos anteriores con una representación visual completa del sistema, incluyendo:
\begin{itemize}
    \item Equipos principales (silo, alimentadores, pre-mezclador, slaker, cámaras de separación, bombas, tanques de almacenamiento).
    \item Flujos de materiales (cal viva, agua, lechada, recirculación).
    \item Ubicación de la instrumentación con sus respectivos tags (FIT, LIT, TIT, PIT).
    \item Notas sobre el consumo de cal y los tiempos de operación.
\end{itemize}
Este PFD es el plano arquitectónico que une los conceptos teóricos y los requisitos de software e instrumentación en una vista unificada del proceso.

\section*{3. Análisis del Código Existente (cal\_monitoring\_backend)}

\subsection*{3.1. ``cal\_monitoring\_backend/main.py''}
Es el punto de entrada de la aplicación FastAPI. Está correctamente inicializado con metadatos del proyecto y un endpoint de prueba (`/`). Los comentarios indican la intención de añadir endpoints para datos de sensores, estado, históricos y configuración.

\subsection*{3.2. ``cal\_monitoring\_backend/core\_logic.py''}
Contiene la lógica de negocio central, incluyendo:
\begin{itemize}
    \item Funciones para cargar datos de sensores de Excel y configuraciones de alarmas de JSON.
    \item Utilidades para mapear TAGs a columnas de Excel.
    \item Constantes para umbrales de alarma (actualmente hardcodeadas, se recomienda externalizarlas).
    \item Funciones para evaluar condiciones de alarma (absolutas, relativas a setpoints, `multiple_and`).
    \item Lógica para determinar el modo de operación actual del sistema.
    \item La clase `ReactivityMonitor`, que implementa la detección y clasificación de curvas de reactividad de la cal basada en cambios de temperatura a lo largo del tiempo.
\end{itemize}
Este archivo es una base sólida con funcionalidades avanzadas ya implementadas.

\subsection*{3.3. ``cal\_monitoring\_backend/test\_core\_logic.py''}
Un script de prueba básico que demuestra la funcionalidad de `core_logic.py`.
\begin{itemize}
    \item Carga datos de ``D:\Cal\Antiguo\Sensor\Tabla\_Completa.xlsx``.
    \item Contenía un 	extbf{error crítico} al intentar cargar la configuración JSON desde ``D:\Cal\Antiguo\Sensor\logica\_planta.py`` (un archivo .py que contenía JSON) en lugar de un .json.
\end{itemize}

\section*{4. Corrección de Ruta de Configuración de Alarmas}
\begin{itemize}
    \item Se identificó que ``D:\Cal\Antiguo\Sensor\logica\_planta.py`` era en realidad un contenido JSON con la configuración de alarmas.
    \item Se creó la carpeta ``cal\_monitoring\_backend/config``.
    \item Se movió el contenido JSON a ``cal\_monitoring\_backend/config/alarm\_config.json``.
    \item Se actualizó la ruta en ``cal\_monitoring\_backend/test\_core\_logic.py`` para cargar correctamente el nuevo archivo JSON.
\end{itemize}
Esta acción asegura una mejor estructura del proyecto y corrige el problema de carga de configuración.

\section*{5. Librerías a Utilizar}
Se han recomendado e instalado las siguientes librerías esenciales para el proyecto:
\begin{itemize}
    \item 	extbf{fastapi}: Framework web.
    \item 	extbf{uvicorn}: Servidor ASGI para FastAPI.
    \item 	extbf{sqlalchemy}: ORM para interacción con base de datos.
    \item 	extbf{pandas}: Manipulación y análisis de datos.
    \item 	extbf{python-dotenv}: Gestión de variables de entorno.
    \item 	extbf{pytest}: Herramienta de testing.
    \item 	extbf{pydantic}: (Dependencia de FastAPI) Validación de datos.
\end{itemize}

\section*{6. Próximos Pasos (TODO List)}
Se ha generado un TODO list detallado para las próximas fases del proyecto, enfocándose en la Etapa 2 y sentando las bases para la Etapa 3.

\end{document}
