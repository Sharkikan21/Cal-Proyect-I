% Template:     Informe/Reporte LaTeX
% Documento:    Archivo principal
% Versión:      5.5.3 (05/09/2018)
% Codificación: UTF-8
%
% Autor: Pablo Pizarro R. @ppizarror
%        Facultad de Ciencias Físicas y Matemáticas
%        Universidad de Chile
%        pablo.pizarro@ing.uchile.cl, ppizarror.com
%
% Manual template: [http://latex.ppizarror.com/Template-Informe/]
% Licencia MIT:    [https://opensource.org/licenses/MIT/]



\documentclass[letterpaper,11pt]{article}
\usepackage[utf8]{inputenc}
\usepackage[spanish]{babel}

\usepackage[T1]{fontenc}
\usepackage{array}


% INFORMACIÓN DEL DOCUMENTO Y DATOS PARA PORTADA
% Para cambiar el estilo de la portada y los encabezados-pie de página entrar a la carpeta lib y el archivo config.tex (lib\config.tex) y modificar las lineas 33 y 32 (funciones \portraitstyle y \hfstyle) cambiando el número del parámetro styleX.
% Importante al cambiar el estilo de portada VERIFICAR que esta cumpla los requisitos formales.
\def\titulodelinforme {Reconstrucción Examen} % <--- modificar ---
\def\temaatratar {} % <--- modificar ---

\def\autordeldocumento {Simón Villavicencio} 
\def\nombredelcurso {Mecanica cuantica}
\def\codigodelcurso {EL6046-1} % <--- modificar si es necesario ---

\def\nombreuniversidad {Universidad de Chile}
\def\nombrefacultad {Facultad de Ciencias Físicas y Matemáticas}
\def\departamentouniversidad {Departamento de fisica}
\def\imagendepartamento {departamentos/fcfm}
\def\imagendepartamentoescala {0.2}
\def\localizacionuniversidad {Santiago, Chile}

% INTEGRANTES, PROFESORES Y FECHAS
% PORTADA (Valor 0,3 puntos)
\def\tablaintegrantes {
\begin{tabular}{ll}
	Estudiante:
	& \begin{tabular}[t]{@{}l@{}}
		 % <--- Modificar ---
            %Ana Rubilar\\
             
		Simón Villavicencio
      
	\end{tabular} \\
 
	%Mesón de trabajo:
	%& \begin{tabular}[t]{@{}l@{}}
	%    N° X        % 
	%\end{tabular}\\
	Profesor:
	& \begin{tabular}[t]{@{}l@{}}
		  %Alex Navas \\
		  Luis Foa. 
	\end{tabular} \\
 
	%Auxiliares:
	%& \begin{tabular}[t]{@{}l@{}}
	%	Alex Rivas\\
%
	%\end{tabular} \\
	%& \\

	%Auxiliar:
	%%	Sebastian Guzman \\
        %Constanza Leiva \\
        %Juan P. Baez H.\\
        %Kevin Chamorro
	%%\end{tabular} \\
	%&% \\

%    Ayudantes:
%	& \begin{tabular}[t]{@{}l@{}}
		%Camila Palma Caris \\
 %       Constanza Leiva \\
  %      Juan P. Baez H.\\
   %     Kevin Chamorro
	%\end{tabular} \\
	& \\
	\multicolumn{2}{l}{Fecha: \today} \\
	\multicolumn{2}{l}{\localizacionuniversidad}
\end{tabular}}{
}
\setlength\parindent{0pt}

% CONFIGURACIONES
\input{lib/config}

% IMPORTACIÓN DE LIBRERÍAS
\input{lib/env/imports}



% IMPORTACIÓN DE FUNCIONES Y ENTORNOS
\input{lib/cmd/all}

% IMPORTACIÓN DE ESTILOS
\input{lib/style/all}

% CONFIGURACIÓN INICIAL DEL DOCUMENTO
\input{lib/cfg/init}

% INICIO DE LAS PÁGINAS
\begin{document}

% PORTADA
\input{lib/page/portrait}

% CONFIGURACIÓN DE PÁGINA Y ENCABEZADOS
\input{lib/cfg/page}



% CONFIGURACIONES FINALES
\input{lib/cfg/final}



% ======================= INICIO DEL DOCUMENTO =======================

% Portada (ya creada, valor de 0.3 puntos)
% Resumen (ya creado, valor de 0.7 puntos)
\noindent
\newpage



% TABLA DE CONTENIDOS - ÍNDICE
\input{lib/page/index} % Índice, se puede borrar





\newpage
% Análisis de resultados (valor de 0.5 puntos)
%\section*{Problema 1: Oscilador Armónico Isotrópico y Efecto Zeeman (Reconstrucción)}

Considere una partícula de masa $m$ y carga $q$ sometida a un potencial armónico tridimensional isotrópico
\begin{equation}
V(r)=\frac{1}{2} m \omega^{2} r^{2}.
\end{equation}

El Hamiltoniano en coordenadas esféricas se escribe como:
\begin{equation}
H_{0} = -\frac{\hbar^{2}}{2m}\left[\frac{1}{r^{2}}\frac{\partial}{\partial r}\left(r^{2}\frac{\partial}{\partial r}\right) - \frac{\hat{L}^{2}}{\hbar^{2} r^{2}}\right] 
+ \frac{1}{2} m \omega^{2} r^{2}.
\end{equation}

Los autovalores de energía para este sistema están dados por
\begin{equation}
E_{N} = \hbar \omega \left( N + \frac{3}{2} \right), \qquad N = 0,1,2,\ldots
\end{equation}
donde $N$ es el número cuántico principal.

\subsection*{1) Degeneración en base Cartesiana}

Determine la degeneración del nivel de energía correspondiente a $N=1$.

\textbf{Hint:} Para determinar la degeneración de manera sencilla, recuerde que el Hamiltoniano puede separarse en coordenadas cartesianas como la suma de tres osciladores independientes en $x,y,z$. Cuente las posibles combinaciones de números cuánticos $(n_x,n_y,n_z)$ que satisfacen la condición de energía para $N=1$.

\subsection*{2) Momento Angular y Paridad}

Dado que el Hamiltoniano es invariante ante rotaciones ($[H_{0},\hat{L}^{2}]=0$), los autoestados pueden caracterizarse por los números cuánticos $(N,l,m)$. Utilizando la relación entre el número cuántico principal y los nodos radiales, o argumentos de paridad, determine qué valores del momento angular orbital $l$ son posibles para el nivel $N=1$.

\textbf{Nota:} Recuerde que la paridad del estado en coordenadas esféricas está determinada por el armónico esférico $Y^{m}_{l}$, teniendo paridad $(-1)^{l}$.

\subsection*{3) Perturbación Magnética (Efecto Zeeman)}

El sistema se coloca ahora en presencia de un campo magnético uniforme y constante en la dirección $z$, 
\begin{equation}
\vec{B} = B_{0} \, \hat{z}.
\end{equation}

La interacción con el campo magnético agrega un término perturbativo al Hamiltoniano debido al momento magnético orbital, dado por:
\begin{equation}
H' = -\vec{\mu}_{L} \cdot \vec{B} = \frac{q B_{0}}{2m} \, \hat{L}_{z},
\end{equation}
donde se ha despreciado el término diamagnético proporcional a $B_{0}^{2}$.

\subsubsection*{a) Escriba el Hamiltoniano total}
\begin{equation}
H = H_{0} + H'.
\end{equation}

\subsubsection*{b) Explique física y matemáticamente por qué este término rompe la degeneración del nivel $N=1$ encontrada en el inciso 1.}

\subsubsection*{c) Calcule las correcciones de energía a primer orden y bosqueje el desdoblamiento de los niveles de energía indicando el valor de $m$ correspondiente a cada nuevo nivel.}

\section*{Problema 2 : Entrelazamiento y Correlaciones de Espín}

Considere un sistema de dos partículas de espín $1/2$ (A y B) preparadas en el \textit{estado singlete} (espín total cero). 
En la base de autoestados de $\hat{S}_z$ ($\lvert +_z \rangle$, $\lvert -_z \rangle$), este estado se escribe como:
\begin{equation}
\lvert \psi_{\text{singlete}} \rangle
= \frac{1}{\sqrt{2}}
\left( \lvert +_z \rangle_A \lvert -_z \rangle_B
      - \lvert -_z \rangle_A \lvert +_z \rangle_B \right).
\end{equation}

Suponga que un observador (Alicia) mide la componente del espín de la partícula A en la dirección $x$ ($\hat{S}_x$) y obtiene el resultado $+ \hbar / 2$. Inmediatamente después, un segundo observador (Beto) mide la componente del espín de la partícula B en la dirección $z$ ($\hat{S}_z$).

Calcule la probabilidad de que Beto obtenga el resultado $+ \hbar / 2$ en su medición.



\section*{Problema 2 : Comportamiento de la función de onda frente a un potencial escalón}

Considere el potencial unidimensional definido por
\[
V(x) =
\begin{cases}
V_0, & 0 < x < a, \\
0, & x > a,
\end{cases}
\]
y suponga que una partícula de energía total \(E\) satisface \(E > V_0\).
La figura adjunta muestra cuatro posibles formas cualitativas para la densidad
de probabilidad \(|\psi(x)|^2\), cada una dibujada sobre el mismo potencial.
 \begin{figure}[!h]
     \centering
     \includegraphics[width=0.5\linewidth]{img/ejemplos/potenciales god asi.png}
     \caption{Potenciales P2}
     \label{fig:placeholder}
 \end{figure}

\begin{enumerate}
    \item Basándose en la ecuación de Schrödinger y en las condiciones de continuidad
    de la función de onda, determine cuál de las cuatro gráficas podría corresponder
    a una solución física para \(\psi(x)\).

    \item Justifique rigurosamente por qué las restantes gráficas no pueden representar
    una solución válida para \(\psi(x)\) en este potencial.

\end{enumerate}


\section*{Problema 2 : Dinámica y Estacionariedad}

En el contexto de la evolución temporal de sistemas cuánticos:

\begin{enumerate}[a)]
    \item Defina rigurosamente qué es un \textit{Estado Estacionario} tanto matemáticamente como físicamente. 
    Mencione dos propiedades fundamentales que caracterizan a estos estados y dé un ejemplo concreto de un sistema físico en dicho estado.
    
    \item Considere ahora la siguiente afirmación hecha por un estudiante:
    \begin{quote}
        ``Dado que el colapso de la función de onda lleva al sistema a un autoestado del operador medido, 
        la densidad de probabilidad
        \begin{equation}
        \lvert \Psi(x,t) \rvert^{2}
        \end{equation}
        de dicho estado permanecerá congelada en el tiempo (estacionaria) mientras no se realicen nuevas mediciones, 
        independientemente de qué observable se haya medido''.
    \end{quote}
    
    ¿Está usted de acuerdo con esta afirmación? Justifique su respuesta utilizando la ecuación de Schrödinger
    \begin{equation}
    i\hbar \frac{\partial}{\partial t} \lvert \Psi(t) \rangle = \hat{H} \lvert \Psi(t) \rangle
    \end{equation}
    y argumentos de conmutación
    \begin{equation}
    [\hat{Q},\hat{H}].
    \end{equation}
\end{enumerate}

\section*{Problema 3, Inciso 1: Resonancias de Transmisión}

\textbf{Enunciado Reconstruido:}

``Considere una partícula de energía $E$ que incide sobre una barrera de potencial rectangular de altura $V_{0}$ y ancho $L$, tal que $E > V_{0}$. Aunque clásicamente la partícula debería atravesar la barrera sin problemas, cuánticamente existe una probabilidad de reflexión no nula. Sin embargo, para ciertos valores específicos de la energía, se observa que el coeficiente de transmisión es exactamente 
\begin{equation}
T = 1
\end{equation}
(transmisión perfecta).

\begin{enumerate}[a)]
    \item Explique físicamente cómo es posible que no haya reflexión ($R = 0$) utilizando conceptos de \textit{interferencia de ondas}.
    \item Mencione la \textit{analogía óptica} correspondiente a este fenómeno físico.
\end{enumerate}
''

\section*{Problema: Semejanza y operadores hermíticos}

Considere la matriz
\[
E =
\begin{pmatrix}
1 & \Delta \\
0 & 1
\end{pmatrix},
\qquad \Delta \in \mathbb{C}.
\]

Dos estudiantes, Abi y Abel, discuten sobre esta matriz:

\begin{itemize}
    \item Abi afirma que es posible encontrar una matriz invertible $S$ tal que
    la matriz transformada
    \[
    E' = S^{-1} E S
    \]
    sea hermítica.

    \item Abel duda de esta afirmación.
\end{itemize}

\begin{enumerate}
    \item Determine quién tiene razón. 

    \item Sea ahora $A$ un operador hermítico arbitrario, es decir,
    $A^\dagger = A$, y considere la transformación de semejanza
    \[
    A' = S^{-1} A S,
    \]
    donde $S$ es una matriz invertible (no necesariamente unitaria).
    Encuentre la condición que debe satisfacer $S$ para que $A'$ resulte
    hermítico, es decir, para que $(A')^\dagger = A'$.

\end{enumerate}



\section*{Problema: Colapso de la Función de Onda y Mediciones}

La figura adjunta muestra la densidad de probabilidad de posición 
$\lvert \Psi(x,y) \rvert^{2}$ para una partícula en el plano $xy$, 
preparada en un estado de paquete de ondas gaussiano en el instante $t = t_{0}$.

Considerando los postulados de la mecánica cuántica respecto a la medición, responda:

\begin{enumerate}[a)]
    \item Bosqueje cualitativamente, en el plano $xy$, cómo se vería la densidad de probabilidad 
    $\lvert \Psi(x,y) \rvert^{2}$ inmediatamente después de realizar una medición precisa 
    de la posición de la partícula.

    \item Bosqueje cualitativamente, en el mismo plano $xy$, cómo se vería la densidad de probabilidad 
    $\lvert \Psi(x,y) \rvert^{2}$ inmediatamente después de realizar una medición precisa 
    del momentum lineal $p$ de la partícula.
\end{enumerate}


\begin{figure}[!h]
    \centering
    \includegraphics[width=0.3\linewidth]{img/ejemplos/Plot_probabilidad.png}
    \caption{Caption}
    \label{fig:placeholder}
\end{figure}



\newpage
% Conclusión (valor 1 punto)

\section*{1. Entendimiento del Flujo de Proceso ("La Historia de la Cal")}

Se ha realizado un análisis detallado del diagrama de flujo de proceso (PFD) `flujo\_mineria.pdf` y se ha correlacionado con la información existente en el código (`cal\_monitoring\_backend/`) y las referencias a archivos previos (`Antiguo/`). Este análisis nos permite establecer una narrativa clara del viaje de la cal a través de la planta, fundamental para la creación de escenarios de prueba realistas.

\subsection*{1.1. Etapas del Proceso}

El proceso de monitoreo de la lechada de cal se descompone en las siguientes etapas principales:

\begin{enumerate}[label=\arabic*.]
    \item \textbf{Almacenamiento de Cal Viva:}
    \begin{itemize}
        \item \textit{Historia:} La cal viva (materia prima) se recibe y almacena en un gran silo.
        \item \textit{Relación con el Proyecto:} El PFD muestra el "Silo de Cal". En el código, los sensores `2270-LIT-11825` (nivel) y las alarmas asociadas (`2270-LSHH-11826`, `2270-LSLL-11829`) se refieren directamente a esta etapa.
    \end{itemize}

    \item \textbf{Dosificación y Alimentación:}
    \begin{itemize}
        \item \textit{Historia:} La cal viva se extrae del silo y se alimenta de manera controlada al equipo de hidratación.
        \item \textit{Relación con el Proyecto:} El control se realiza mediante el "Alimentador de Tornillo" (`2270-SAL-11817`) y la "Válvula Rotatoria" (`2270-SAL-11818`), cuyas señales son utilizadas en `data\_generator.py` para simular la alimentación de cal.
    \end{itemize}

    \item \textbf{Hidratación (El "Apagado" de la Cal):}
    \begin{itemize}
        \item \textit{Historia:} La cal viva reacciona exotérmicamente con agua en un "Slaker" o "Apagador" para formar hidróxido de calcio (lechada de cal). Esta es la etapa central del proceso.
        \item \textit{Relación con el Proyecto:} El PFD muestra el equipo de mezcla. Los sensores `2270-FIT-11801` (flujo de agua) y `2270-TT-11824A/B` (temperaturas del slaker) son críticos aquí. La clase `ReactivityMonitor` en `core\_logic.py` analiza la curva de temperatura del sensor `2270-TT-11824B`, implementando directamente el concepto de \textit{reactividad} explicado en los libros de Guillermo.
    \end{itemize}

    \item \textbf{Separación y Clasificación:}
    \begin{itemize}
        \item \textit{Historia:} La lechada de cal se procesa para remover impurezas y partículas gruesas ("arenilla" o "grit") en una cámara de separación o hidrociclones.
        \item \textit{Relación con el Proyecto:} El PFD muestra los hidrociclones. Los sensores `2270-LIT-11850` (nivel de la cámara de separación) y alarmas como `2270-PALL-11834` (presión del hidrociclón) en `alarm\_config.json` confirman la relevancia de esta etapa para el monitoreo.
    \end{itemize}

    \item \textbf{Almacenamiento y Distribución de Lechada Final:}
    \begin{itemize}
        \item \textit{Historia:} La lechada de cal lista para su uso se almacena en tanques y se bombea a los puntos de aplicación en la minera.
        \item \textit{Relación con el Proyecto:} El PFD finaliza con tanques de lechada y bombas. Sensores como `2270-LIT-11845` (nivel de tanque de descarga) y variables de monitoreo de bombas y sistemas de lubricación (`2270-PIT-11895`, `2270-TIT-11893`, `2270-FSL-11896`) están asociados a esta etapa.
    \end{itemize}
\end{enumerate}

\section*{2. Relación Clave con los Libros de Guillermo y el Diseño del Software}

La comprensión profunda de los libros de Guillermo Coloma Álvarez es \textbf{esencial} para desarrollar un software que no solo monitoree, sino que interprete y optimice el proceso de la cal de manera inteligente. Los libros no son solo teoría; proporcionan el marco conceptual para validar y enriquecer nuestro modelo de simulación y lógica de control.

\begin{itemize}
    \item \textbf{La Reactividad de la Cal como Fundamento:} Los libros de Guillermo enfatizan que la \textit{reactividad de la cal} (definida por la velocidad y magnitud del aumento de temperatura durante el apagado) es una propiedad crítica que determina su eficiencia.
    \begin{itemize}
        \item \textit{Conexión con Datos:} Nuestro sensor `2270-TT-11824B` (temperatura del slaker) mide directamente esta reacción. Un simulador avanzado debe poder generar curvas de temperatura que varíen significativamente según la \textit{calidad de cal simulada} (ej. una cal "dura de apagar" o de baja reactividad mostrará una curva más lenta y menos intensa, como se describe en los Gráficos de Reactividad de los libros).
        \item \textit{Impacto en el Software:} La clase `ReactivityMonitor` de `core\_logic.py` está perfectamente posicionada para clasificar estas curvas, pero su eficacia depende de que el simulador le proporcione datos que reflejen las distintas calidades de cal.
    \end{itemize}

    \item \textbf{La Calidad de la Cal y el "CaO Equivalente":} Los libros distinguen entre `CaO libre`, `CaO crudo` y `CaO combinado/requemado`, todos contribuyentes al `CaO Equivalente` y a la capacidad alcalinizante total. La presencia de impurezas impacta directamente en estas proporciones.
    \begin{itemize}
        \item \textit{Conexión con Datos:} Actualmente, nuestro simulador genera solo valores de sensores. Para reflejar la riqueza de los libros, el simulador debe permitir definir el \textbf{perfil de calidad de la cal de entrada} (ej. 90\% CaO libre, 5\% impurezas). Estos parámetros, aunque no son directamente sensores, deben influir en el comportamiento de los sensores simulados (ej. un mayor porcentaje de `CaO crudo` podría generar lecturas de temperatura de apagado anómalas o más lentas).
        \item \textit{Impacto en el Software:} `core\_logic.py` podría entonces calcular el "CaO Equivalente" simulado, y este valor, junto con la reactividad, sería un indicador clave de rendimiento de la cal procesada, directamente extraído de la teoría de Guillermo.
    \end{itemize}

    \item \textbf{Impurezas, Agua y Eventos Anormales:} Los libros detallan cómo las impurezas de la caliza o del agua, así como la temperatura o la dosificación incorrecta de agua, pueden llevar a problemas como la formación de "arenillas", incrustaciones, menor reactividad o consumo excesivo de energía.
    \begin{itemize}
        \item \textit{Conexión con Datos:} Nuestros escenarios de prueba deben contemplar estas situaciones. Por ejemplo, simular un "exceso de impurezas en el agua" podría generar datos de sensores que gradualmente lleven a una alarma de "incrustación" (si desarrollamos una lógica para ello), o un escenario de "cal de baja calidad" resultaría en una reactividad pobre que el software debería detectar.
        \item \textit{Impacto en el Software:} Esto nos guiará para crear lógica de alarmas y monitoreo más sofisticada en `core\_logic.py` que considere estas interacciones complejas, tal como se detalla en los diagramas de control de las plantas de lechada de los libros (ej. el diagrama de calidad de la lechada en la página 117 de "CaO más alto implica ahorro...").
    \end{itemize}
\end{itemize}

Esta integración de la teoría de Guillermo con los datos simulados permitirá que nuestro software no solo sea funcional, sino que también actúe como una herramienta de aprendizaje y optimización basada en un conocimiento profundo del proceso de la cal.

\section*{3. Avances de la Sesión Actual (17 de Febrero de 2026)}

Durante esta sesión, se han logrado los siguientes avances significativos:

\begin{enumerate}[label=\arabic*.]
    \item \textbf{Recopilación de Contexto Inicial:} Se realizó una revisión exhaustiva de todos los archivos del proyecto, incluyendo `resumen\_progreso.txt`, las bitácoras anteriores y el código base en `cal\_monitoring\_backend/`.

    \item \textbf{Análisis Profundo de la Documentación Clave:} Se leyeron y analizaron los libros de Guillermo Coloma Álvarez, "La Cal ¡Es un Reactivo Químico!" y "CaO más alto implica ahorro de energía y agua", extrayendo insights críticos sobre la química de la cal, su reactividad, impurezas y la optimización del proceso. Este análisis ha sido fundamental para comprender la "Historia de la Cal" y las bases conceptuales del proyecto.

    \item \textbf{Diagnóstico y Solución del Problema de Ejecución de la API:} Se identificó la causa del problema reportado ("imagen en blanco" en el navegador) como una ejecución incorrecta de la aplicación FastAPI. Se proporcionaron instrucciones claras sobre cómo ejecutarla correctamente usando `uvicorn cal\_monitoring\_backend.main:app --reload` desde la raíz del proyecto, asegurando que las importaciones relativas funcionen como se espera.

    \item \textbf{Definición Detallada de la "Historia de la Cal":} Se desglosó el flujo de proceso de la planta de cal en 5 etapas principales (Almacenamiento de Cal Viva, Dosificación y Alimentación, Hidratación, Separación y Clasificación, Almacenamiento y Distribución de Lechada Final). Para cada etapa, se identificaron los equipos clave del PFD (`flujo\_mineria.pdf`), los TAGs de sensores relevantes del código existente (`alarm\_config.json`, `data\_generator.py`) y su propósito funcional, creando una narrativa coherente del proceso.

    \item \textbf{Relación Crítica de los Libros con el Diseño del Software:} Se estableció y documentó explícitamente cómo los conceptos de los libros de Guillermo (como la reactividad de la cal, el CaO equivalente, el impacto de las impurezas y la calidad del agua) deben ser integrados en el diseño del software, particularmente en la generación de datos de prueba y la lógica de negocio. Se enfatizó que estos conceptos son cruciales para crear un modelo de simulación realista y una lógica de monitoreo inteligente.

    \item \textbf{Generación de Datos de Prueba para la Etapa 1 (Almacenamiento de Cal Viva):}
    \begin{itemize}
        \item Se definieron 5 micro-escenarios específicos para la simulación del nivel del silo y sus alarmas asociadas (`2270-LIT-11825`, `2270-LSHH-11826`, `2270-LSLL-11829`): nivel estable, vaciándose, llenándose, alarma de nivel alto y alarma de nivel bajo.
        \item Se proporcionó un prompt detallado para un asistente de código IA para generar un script de Python (`scenario\_generator.py`) que produce estos 5 escenarios en archivos CSV individuales.
        \item Se confirmó la exitosa generación del script `scenario\_generator.py` y los cinco archivos CSV correspondientes dentro de la carpeta `cal\_monitoring\_backend/`.
    \end{itemize}
\end{enumerate}

Los avances realizados en esta sesión han sentado una base sólida para el desarrollo futuro, asegurando que el software se construya con una comprensión profunda del proceso y una alineación directa con los principios fundamentales establecidos en la documentación del proyecto y los libros de referencia.


\newpage


\section{Introducción y Objetivos}

El presente informe detalla el progreso, los análisis realizados y los desarrollos implementados en el proyecto de software para el \textbf{monitoreo del flujo y proceso de la cal} en una operación minera.

El objetivo principal es construir un sistema de software robusto que permita la supervisión integral y el análisis operacional del sistema de preparación y distribución de lechada de cal. Esto se alinea con los requisitos de la \textbf{Etapa 2 (Supervisión Integral)} y sienta las bases para la futura \textbf{Etapa 3 (Control Avanzado y Optimización)}.

\section{Análisis de Documentación Fundamental}

Se realizó un análisis exhaustivo de la documentación técnica proporcionada, extrayendo el contexto esencial para el diseño y desarrollo del software.

\subsection{La Cal como Reactivo Químico (Libro de G. Coloma)}
Este texto proporcionó la base teórica sobre la química de la cal, destacando la importancia crítica de la \textbf{reactividad}, la hidratación (apagado) y las variables de proceso (temperatura, calidad del agua, etc.) que afectan la calidad final de la lechada.

\subsection{Especificación de Instrumentación y Software}
Este documento es la hoja de ruta del proyecto. Define las \textbf{Etapas 2 y 3}, detallando los requisitos funcionales como dashboards en tiempo real, registro histórico, sistema de alarmas y la necesidad de monitorear la instrumentación existente (LT, FT, pHT, DT).

\subsection{Filosofía de Control}
Proporciona el ``cómo'' se controla la planta actualmente. Se extrajeron tablas de \textbf{alarmas e interlocks}, secuencias operacionales, y los \textbf{TAGs} específicos de la instrumentación del Sistema de Control Distribuido (DCS), que son la base para nuestra configuración de alarmas.

\subsection{Diagrama de Flujo de Proceso (PFD)}
El PFD (`flujo\_mineria.pdf`) ofreció una representación visual completa de la planta, permitiendo unir los conceptos teóricos y los requisitos de software en una vista arquitectónica unificada del proceso físico.

\section{Estructura y Lógica del Software Actual}

El núcleo del proyecto reside en la carpeta `cal\_monitoring\_backend/`, que contiene una aplicación Python moderna basada en FastAPI.

\begin{itemize}
    \item \textbf{`main.py`}: Es el punto de entrada de la \textbf{API REST}. Está diseñado para exponer los datos y la lógica del sistema a través de endpoints, como `/api/v1/status`, permitiendo que una futura interfaz de usuario (frontend) consuma la información en tiempo real.
    
    \item \textbf{`core\_logic.py`}: Es el cerebro de la aplicación. Contiene la lógica de negocio para:
    \begin{itemize}
        \item Cargar configuraciones y datos.
        \item \textbf{Evaluar un sistema de alarmas complejo} basado en condiciones absolutas, relativas a setpoints y lógicas (`multiple\_and`).
        \item Determinar el \textbf{modo de operación} de la planta (ej.\ ``produciendo'', ``lavando'', ``inactivo'').
        \item La clase \textbf{`ReactivityMonitor`}, una implementación clave inspirada en los libros de Guillermo Coloma, diseñada para detectar y clasificar las curvas de reactividad de la cal basándose en la evolución de la temperatura.
    \end{itemize}
    
    \item \textbf{`config/alarm\_config.json`}: Archivo de configuración que \textbf{externaliza toda la lógica de alarmas}. Contiene los TAGs de los sensores, los equipos asociados, los umbrales y las descripciones de cada condición de alarma, replicando la filosofía de control del DCS.
    
    \item \textbf{`data\_generator.py` y `scenario\_generator.py`}: Scripts de Python dedicados a la \textbf{simulación de datos de sensores}. Permiten generar escenarios realistas y dinámicos para probar la lógica del backend y facilitar el desarrollo del frontend sin depender de datos de una planta real.
\end{itemize}

\section{Avances Significativos y Decisiones Clave}

\subsection{Definición de la ``Historia de la Cal''}
Se ha establecido una narrativa clara y detallada del proceso de la cal, correlacionando el PFD con los conceptos de los libros de Guillermo Coloma y los TAGs de los sensores del proyecto. Este entendimiento es fundamental para crear simulaciones y lógicas de monitoreo que reflejen fielmente la realidad operativa. El proceso se ha dividido en 5 etapas:
\begin{enumerate}
    \item Almacenamiento de Cal Viva (Silo de Cal).
    \item Dosificación y Alimentación (Alimentador de Tornillo).
    \item Hidratación o ``Apagado'' (Slaker).
    \item Separación y Clasificación (Hidrociclones).
    \item Almacenamiento y Distribución de Lechada Final.
\end{enumerate}

\subsection{Integración de Conocimiento Experto}
Se ha documentado explícitamente cómo los conceptos de los libros de Guillermo Coloma (reactividad, CaO equivalente, impurezas) deben guiar el desarrollo. El software no solo debe monitorear, sino \textbf{interpretar el proceso} a la luz de esta teoría, permitiendo una optimización futura.

\subsection{Desarrollo de un Simulador de Escenarios}
Reconociendo la necesidad de datos dinámicos para el desarrollo y las pruebas, se ha priorizado y completado la creación de un generador de escenarios.
\begin{itemize}
    \item Se creó el script \textbf{`scenario\_generator.py`}.
    \item Este script genera archivos \textbf{CSV} con datos que simulan 5 micro-escenarios para el nivel del silo de cal:
    \begin{enumerate}
        \item Nivel estable.
        \item Vaciándose (en producción).
        \item Llenándose (recargando).
        \item Alarma de nivel alto.
        \item Alarma de nivel bajo.
    \end{enumerate}
    \item Estos escenarios son la primera implementación del generador de datos y sirven como base para futuras simulaciones más complejas.
\end{itemize}

\subsection{Aclaración de Requisitos Futuros}
Se ha establecido que la arquitectura a mediano plazo deberá migrar hacia el uso de servicios en la nube de \textbf{Amazon Web Services (AWS)} para la persistencia y gestión de datos (ej. S3, RDS, Kinesis), reemplazando las soluciones temporales basadas en archivos.

\section{Próximos Pasos}

El plan de acción inmediato se centra en expandir la funcionalidad de la API y la simulación:
\begin{enumerate}
    \item \textbf{Expandir la API (`main.py`):} Desarrollar y refinar los endpoints de la API para que consuman los datos generados por el simulador y expongan de manera clara el estado de la planta, las alarmas activas y las curvas de reactividad.
    
    \item \textbf{Integrar Escenarios Generados:} Modificar el simulador principal (`data\_generator.py`) para que pueda leer y utilizar los archivos CSV generados por `scenario\_generator.py`, permitiendo ejecutar pruebas controladas desde la API.
    
    \item \textbf{Ampliar el Generador de Escenarios:} Crear nuevos escenarios de simulación que cubran otras etapas del proceso, como las curvas de reactividad (alta, media, baja) y condiciones de alarma en el Slaker.
    
    \item \textbf{Iniciar Desarrollo de Interfaz de Usuario (Frontend):} Con una API funcional que provee datos dinámicos, se puede comenzar el desarrollo de una página web que visualice en tiempo real el estado de la planta, las alarmas y los gráficos de tendencia.
\end{enumerate}



\newpage
%% !TEX root = main.tex
\section*{Mapeo de Sensores: La Historia de la Cal}

Este documento establece la relación explícita entre el listado definitivo de sensores y las 5 etapas de la ``Historia de la Cal''. Este mapeo es la base para la lógica de simulación, la configuración de alarmas y el desarrollo de la interfaz de usuario.

\begin{longtable}{|l|p{6cm}|l|l|}
\hline
\textbf{TAG} & \textbf{Descripción} & \textbf{Tipo} & \textbf{Rol en la Historia} \\ \hline
\endfirsthead
\hline
\textbf{TAG} & \textbf{Descripción} & \textbf{Tipo} & \textbf{Rol en la Historia} \\ \hline
\endhead

\multicolumn{4}{|l|}{\textit{\textbf{Etapa 1: Almacenamiento de Cal Viva (Silo)}}} \\ \hline
2270-LIT-11825 & Transmisor de Nivel Radar Silo & AI (\%) & Inventario principal \\ \hline
2270-LSHH-11826 & Alarma Nivel Muy Alto (Switch) & DI & Protección rebalse \\ \hline
2270-LSLL-11829 & Alarma Nivel Muy Bajo (Switch) & DI & Protección vacío \\ \hline
2270-YL-11826 & Indicador de Nivel (70\%) & DI & Control de recarga \\ \hline
2270-PDAH-11827 & Presión Dif. Filtro de Mangas & AI & Salud del sistema filtrado \\ \hline

\multicolumn{4}{|l|}{\textit{\textbf{Etapa 2: Dosificación y Alimentación}}} \\ \hline
2280-WI-01769 & Pesómetro de Cal (Tph) & AI & Flujo de masa de entrada \\ \hline
2270-SAL-11817 & Alimentador de Tornillo (Estado) & DI & Actuador de carga \\ \hline
2270-SAL-11818 & Válvula Rotatoria Silo (Estado) & DI & Actuador de descarga \\ \hline
2270-SE-11817 & Velocidad Cero Tornillo & DI & Confirmación mecánica \\ \hline

\multicolumn{4}{|l|}{\textit{\textbf{Etapa 3: Hidratación o ``Apagado'' (Vortex/Slaker)}}} \\ \hline
2270-FIT-11801 & Flujo Agua a Apagado & AI & Reactivo de hidratación \\ \hline
2270-TT-11824A & Temperatura Slaker A & AI & Control térmico \\ \hline
2270-TT-11824B & Temperatura Slaker B & AI & \textbf{Cálculo de Reactividad} \\ \hline
2270-PALL-11834 & Presión de Agua Vortex & AI & Eficiencia de pre-mezcla \\ \hline
2270-TAHH-11801 & Alarma Temp. Muy Alta Vortex & DI & Seguridad operativa \\ \hline
2270-ZM-009-06 & Motor Principal Slaker & DI & Estado de proceso \\ \hline

\multicolumn{4}{|l|}{\textit{\textbf{Etapa 4: Separación y Clasificación}}} \\ \hline
2270-LIT-11850 & Nivel Cámara de Separación & AI (\%) & Control de desbaste \\ \hline
2270-LALL-11850 & Alarma Nivel Muy Bajo Cámara & DI & Protección de agitación \\ \hline
2270-ZM-009-31 & Agitador Cámara Separación 1 & DI & Homogeneización \\ \hline

\multicolumn{4}{|l|}{\textit{\textbf{Etapa 5: Almacenamiento y Distribución Final}}} \\ \hline
2270-LIT-11845 & Nivel Caja Bomba Descarga & AI (\%) & Pulmón de distribución \\ \hline
2270-PIT-11895 & Presión Sistema Lubricación & AI & Salud mecánica bombas \\ \hline
2270-FSL-11896 & Flujo Aceite Lubricación & AI & Salud mecánica bombas \\ \hline
2270-TIT-11893 & Temperatura Aceite T1 & AI & Salud mecánica bombas \\ \hline
DT-2270-HDR & Densidad de Lechada & AI & Calidad del producto final \\ \hline
pHT-2270-RGH & pH en Flotación Rougher & AI & Control de proceso final \\ \hline

\end{longtable}

\textbf{Conclusión:} Esta estructura permite que el backend evalúe la salud del sistema no solo por sensor individual, sino por la continuidad lógica del flujo: si el Silo (E1) baja, el Tornillo (E2) debe estar ON, la Temperatura (E3) debe subir y la Densidad (E5) debe estabilizarse.
\newpage
%\section{Pregunta 3}



\section*{Problema 1: Homenaje a la `Matrizenmechanik'. (Puntaje: 24/60)}

Inspirados por el formalismo desarrollado por Heisenberg, Born y Jordan en la década de 1920, consideremos un sistema físico de dimensión 3. En una base ortonormal $\{|v_1\rangle, |v_2\rangle, |v_3\rangle\}$, el Hamiltoniano $H$ y un observable $A$ están dados por:

\begin{equation}
H = E_0 \begin{bmatrix} 1 & 0 & 0 \\ 0 & 0 & 1 \\ 0 & 1 & 0 \end{bmatrix}, \quad A = a \begin{bmatrix} 0 & 1 & 0 \\ 1 & 0 & 0 \\ 0 & 0 & 1 \end{bmatrix}
\end{equation}

donde $E_0$ y $a$ son constantes reales positivas con dimensiones apropiadas.

\begin{enumerate}
    \item[a)] Si se mide la energía, ¿qué valores pueden obtenerse? Indique la degeneración de cada autovalor.
    \item[b)] Suponga que se mide la energía y se obtiene el valor $E_0$. Inmediatamente después se mide $A$. ¿Qué valores se pueden obtener y con qué probabilidades?
    \item[c)] Calcule el conmutador $[H, A]$. ¿Son compatibles estos observables? Comente sobre la existencia de una base común de autoestados.
    \item[d)] Supongamos que al tiempo $t=0$ el sistema se encuentra en el estado:
    \begin{equation}
    |\psi(0)\rangle = \frac{1}{\sqrt{2}} (|v_1\rangle + |v_2\rangle)
    \end{equation}
    Calcule el estado al tiempo $t$, $|\psi(t)\rangle$, y el valor de expectación del observable $A$ en función del tiempo, $\langle A \rangle(t)$.
\end{enumerate}

\section*{Problema 2: Preguntas generales de desarrollo breve. (Puntaje: 18/60)}

\begin{enumerate}
    \item[a)] Enuncie la condición para que un observable $\hat{Q}$ (que no depende explícitamente del tiempo) sea una constante de movimiento. Si $\hat{Q}$ no conmuta con el Hamiltoniano $\hat{H}$, ¿es posible que el valor de expectación $\langle \hat{Q} \rangle$ sea constante en el tiempo para algún estado particular? Justifique.
    
    \item[b)] \textbf{¡Falla en el confinamiento!} Una partícula está en el estado fundamental de una caja infinita (entre $x=0$ y $x=L$). Súbitamente, ocurre una falla en el sistema de confinamiento y la pared derecha se desplaza instantáneamente a $x=2L$. Calcule la probabilidad de que la partícula permanezca en el estado fundamental de la nueva configuración expandida.
    
    \item[c)] \textbf{Diseño de un punto cuántico.} Un estudiante de ingeniería intenta diseñar un dispositivo modelándolo con el siguiente potencial 1D, buscando confinar una partícula y modificar sus propiedades mediante una impureza atractiva en el centro:
    \begin{equation}
    V(x) = \begin{cases} -V_0 \delta(x) & \text{si } -L/2 < x < L/2 \\ \infty & \text{si } |x| \geq L/2 \end{cases}
    \end{equation}
    donde $V_0 > 0$. Bosqueje cualitativamente la función de onda del estado fundamental ($\psi_1(x)$) y del primer estado excitado ($\psi_2(x)$). Justifique las características principales de sus gráficos (simetría, número de nodos, y comportamiento en $x=0$). Notar que se pide bosquejar, no calcular la solución.
\end{enumerate}

\section*{Problema 3: Preguntas generales de desarrollo breve parte 2. (Puntaje: 18/60)}

\begin{enumerate}
    \item[a)] \textbf{La Preocupación de Schrödinger.} Erwin Schrödinger estaba profundamente incómodo con la idea de las superposiciones cuánticas, lo que lo llevó a formular su famosa paradoja del gato: un experimento mental donde un gato estaría simultáneamente vivo y muerto según las reglas cuánticas, algo claramente absurdo en el mundo macroscópico. En el formalismo estándar (interpretación de Copenhague), ¿cuál es el rol del postulado de la medición y por qué es necesario para reconciliar la evolución unitaria (ecuación de Schrödinger) con la observación de resultados definidos en experimentos macroscópicos (¡como un gato vivo o muerto!)?
    
    \item[b)] Considere dos observables, $A$ y $B$, que conmutan ($[A, B] = 0$). Un estudiante afirma: \textit{``Si medimos A y obtenemos el autovalor $a_i$, el estado resultante del sistema es necesariamente un autoestado de B''}. ¿Es esta afirmación siempre verdadera? Justifique su respuesta y, si es falsa, indique bajo qué condición adicional sería verdadera.
    
    \item[c)] En el casino de la FCFM, Abi y Ben discuten qué ocurre tras medir un observable $\Omega$. Han medido $\Omega$ y obtuvieron $\omega_n$, por lo que el sistema quedó en $|\omega_n\rangle$ (autovector de $\Omega$). Ben afirma: \textit{``Como no haremos más mediciones, $|\omega_n\rangle$ seguirá siendo autovector de $\Omega$ para cualquier $t > 0$. Al fin y al cabo, solo evoluciona unitariamente.''} Abi duda: \textit{``No siempre. Puede `desparramarse' en la base de $\Omega$.''}
    
    \textbf{Pregunta (breve):} ¿Quién tiene razón? Justifique en pocas líneas usando la evolución $|\psi(t)\rangle = e^{-iHt/\hbar}|\omega_n\rangle$. Indique la condición (en términos de conmutadores) bajo la cual $|\omega_n\rangle$ permanece autovector de $\Omega$, y explique qué ocurre en caso contrario en el espacio de estados.
\end{enumerate}




\section*{Problema 1: Oscilador armónico con un twist. (18 puntos)}

Considere una partícula de masa $m$ en un potencial armónico de frecuencia $\omega$. Al tiempo $t=0$, el sistema se encuentra en el estado:

\begin{equation}
|\psi(0)\rangle = \frac{1}{\sqrt{2}}|0\rangle + \frac{1}{\sqrt{2}}|2\rangle
\end{equation}

\begin{enumerate}
    \item[a)] Calcule $\langle X^2 \rangle$ y $\langle P^2 \rangle$ al tiempo $t=0$ usando los operadores de creación y destrucción. Verifique que se satisface el principio de incertidumbre.
    \item[b)] Determine la paridad del estado inicial $|\psi(0)\rangle$. Luego, usando argumentos de simetría y sin calcular explícitamente la evolución temporal, demuestre que $\langle X \rangle(t) = 0$ y $\langle P \rangle(t) = 0$ para todo $t > 0$.
    \item[c)] Si al tiempo $t=0$ se aplica instantáneamente una perturbación que desplaza el mínimo del potencial a la posición $x_0$. El nuevo Hamiltoniano es $H' = \frac{P^2}{2m} + \frac{1}{2}m\omega^2(X-x_0)^2$. Sin resolver la ecuación de Schrödinger explícitamente, argumente cualitativamente cómo evolucionará el sistema. ¿Qué valores de expectación oscilarán y con qué frecuencia?
\end{enumerate}

\section*{Problema 2: El retorno del momento angular. (18 puntos)}

Un sistema cuántico tiene momento angular $\ell = 1$ y su Hamiltoniano está dado por:

\begin{equation}
H = \alpha(L_x^2 - L_y^2)
\end{equation}

donde $\alpha > 0$ es una constante con dimensiones apropiadas.

\begin{enumerate}
    \item[a)] Sin hacer cálculos explícitos, explique por qué $H$ debe ser diagonal por bloques en la base común a $L^2$ y $L_z$. ¿Cuántos bloques hay y de qué dimensión?
    \item[b)] Encuentre los autovalores de energía. Ayuda: Puede ser útil reescribir el Hamiltoniano en términos de $L_\pm = L_x \pm iL_y$.
    \item[c)] Si al tiempo $t=0$ el sistema está en el estado $|1, 0\rangle_z$, calcule la probabilidad de encontrar $L_z = \hbar$ al tiempo $t$.
\end{enumerate}

\section*{Problema 3: Preguntas conceptuales/ de desarrollo breve. (24 puntos)}

\begin{enumerate}
    \item[a)] \textbf{Peculiaridad de la reversión temporal.}
    Para una partícula sin spin, el operador de reversión temporal tiene la forma $\Theta = UK$ donde $K$ es el operador de conjugación compleja y $U$ es un operador unitario apropiado. En simetrías continuas, un operador unitario $G(\epsilon) = e^{i\epsilon \hat{O}/\hbar}$ que conmuta con el Hamiltoniano genera una cantidad conservada $\hat{O}$ según el Teorema de Noether cuántico.
    \begin{enumerate}
        \item[i)] Demuestre que si el Hamiltoniano $H$ es invariante bajo reversión temporal (i.e., $\Theta H \Theta^{-1} = H$), entonces se cumple:
        \begin{equation}
        \Theta U(t) \Theta^{-1} = U(-t)
        \end{equation}
        donde $U(t) = e^{-iHt/\hbar}$ es el operador de evolución temporal.
        \item[ii)] Explique por qué esta relación implica que no existe una cantidad conservada asociada a la invariancia bajo reversión temporal, a diferencia de lo que ocurre con simetrías continuas como traslación espacial (momento) o rotación (momento angular).
    \end{enumerate}

    \item[b)] \textbf{Principio variacional.} Un estudiante utiliza el principio variacional con la función de prueba $\psi(x) = A x e^{-\alpha x^2}$ (pensando en estados ligados) para un potencial par $V(x) = V(-x)$. Obtiene una cota de energía $E_{var}$.
    \begin{itemize}
        \item ¿Esta cota corresponde al estado fundamental del sistema? Justifique usando argumentos de paridad.
        \item Si el estudiante hubiera usado en cambio $\phi(x) = A e^{-\beta x^2}$, ¿qué estado estaría estimando? ¿Cuál de las dos cotas será menor y por qué?
    \end{itemize}

    \item[c)] \textbf{Mediciones y evolución temporal.} Considere un sistema preparado en un estado que es combinación de dos autoestados del Hamiltoniano con energías $E_1$ y $E_2$. Se realiza una secuencia de mediciones de energía, con intervalos de tiempo $\Delta t$ entre mediciones sucesivas.
    \begin{itemize}
        \item Para $\Delta t \to 0$, ¿qué ocurre con la evolución del sistema?
        \item Para $\Delta t \gg \hbar/|E_1 - E_2|$, bosqueje cualitativamente la probabilidad de obtener $E_1$ en función del tiempo entre la primera medición (en $t=0$) y una medición posterior.
    \end{itemize}
\end{enumerate}







\newpage
%\input{conclusiones}

%\section{Desarrollo:}
\subsection{¿Qué es BIM?}
Building Information Modeling (BIM) es una metodología de trabajo colaborativa que utiliza modelos digitales 3D para concentrar, en un solo entorno, la información geométrica y alfanumérica de un proyecto de arquitectura, ingeniería y construcción a lo largo de todo su ciclo de vida: diseño, construcción, operación y eventual demolición. Un modelo BIM no es sólo una “maqueta 3D”; cada elemento (muros, losas, equipos, tuberías, caminos, estructuras, etc.) está asociado a propiedades como materiales, dimensiones, costos, vida útil, desempeño energético y vínculos a documentos externos.\\

En la practica, BIM agrega diferentes $"dimensiones"$ al modelo:

\begin{itemize}
    \item 3D: geometría y coordinación espacial.
    \item 4D: programación de obra (tiempo y secuencia constructiva).
    \item 5D: costos y presupuestos vinculados a cantidades automáticas.
    \item 6D y más: información de sostenibilidad, impactos ambientales, mantenimiento, huella de carbono, entre otros.
\end{itemize}

El modelo se gestiona en un entorno común de datos (CDE) que permite a los distintos actores (mandantes, proyectistas, constructores, operadores y autoridades) trabajar sobre una base de información única y trazable, reduciendo errores por incoherencias entre planos, memorias y planillas.\\

En Chile, desde 2016 el programa Planbim Corfo impulsa la adopción de BIM en proyectos públicos, generando el Estándar BIM para Proyectos Públicos y guías de implementación.

 A partir de 2020 diversos mandantes del Estado han empezado a exigir la entrega de modelos BIM en licitaciones, y actualmente (2025) existe una Hoja de Ruta BIM que busca alcanzar cerca de un 70 \% de adopción de BIM al 2028, especialmente en edificación pública e infraestructura.

\newpage
\subsection{¿Cómo aplicar BIM en el desarrollo de una EIA?}

La aplicación de BIM en el desarrollo de un Estudio de Impacto Ambiental (EIA) consiste en usar el modelo digital del proyecto como ``columna vertebral'' de la información técnica y ambiental. En lugar de elaborar el EIA sólo a partir de planos 2D, memorias y planillas desconectadas, BIM permite integrar en un mismo entorno la geometría del proyecto, los datos del entorno físico y las variables ambientales relevantes, manteniendo coherencia y trazabilidad a lo largo de todo el ciclo del estudio.

En una visión general, BIM puede apoyar las distintas partes del EIA de la siguiente manera:

\begin{itemize}
    \item \textbf{Formulación del proyecto y alternativas.}  
    En la etapa temprana se modelan en BIM las distintas alternativas de emplazamiento, trazado o configuración del proyecto (por ejemplo, variantes de una carretera, de una línea eléctrica o de un edificio). Cada alternativa puede asociarse a información territorial y normativa (uso de suelo, áreas protegidas, servidumbres, distancias a receptores sensibles), lo que facilita comparar sus implicancias ambientales antes de definir la alternativa a someter al SEIA.

    \item \textbf{Descripción del entorno y línea base.}  
    El modelo BIM puede vincularse con información geoespacial (GIS), levantamientos topográficos, catastros de vegetación, hidrología, equipamientos existentes, etc. De este modo, el entorno relevante se representa tridimensionalmente (relieve, cuerpos de agua, edificaciones cercanas) y se le agregan atributos ambientales (calidad de aire, niveles de ruido de fondo, sensibilidad ecológica), lo que mejora la comprensión espacial de la línea base.

    \item \textbf{Identificación y evaluación de impactos.}  
    A partir del modelo 3D--4D se pueden simular y cuantificar impactos: volúmenes de movimiento de tierras, ocupación de fajas, áreas de intervención de vegetación, sombras proyectadas, visibilidad desde ciertos puntos de observación, proximidad a cursos de agua, entre otros. El carácter paramétrico de BIM hace que cualquier cambio de diseño se actualice automáticamente en las métricas usadas en el EIA (superficies afectadas, volúmenes, distancias), reduciendo errores de cálculo y de coherencia entre capítulos.

    \item \textbf{Diseño de medidas de mitigación, reparación y compensación.}  
    Las medidas ambientales (pantallas acústicas, franjas de reforestación, drenajes, cierres perimetrales, pasos de fauna, etc.) pueden modelarse como elementos BIM con propiedades específicas (dimensiones, materiales, eficiencia acústica o hidráulica, costos). Esto permite verificar visual y cuantitativamente su efectividad, chequear interferencias con otras disciplinas y asegurar el cumplimiento de exigencias normativas o de los términos de referencia.

    \item \textbf{Programación y costos de la gestión ambiental (4D y 5D).}  
    Asociando el modelo a un cronograma (4D) y a datos de costos (5D), es posible planificar en detalle cuándo se ejecuta cada actividad con implicancias ambientales (despeje de vegetación, excavaciones, manejo de residuos, implementación de medidas de control) y cuánto cuesta. Esto facilita elaborar el Plan de Manejo Ambiental, estimar recursos necesarios y respaldar la viabilidad económica de las medidas propuestas.

    \item \textbf{Participación ciudadana y comunicación de impactos.}  
    El modelo BIM permite generar visualizaciones comprensibles para actores no técnicos (recorridos virtuales, imágenes 3D, animaciones de la fase de construcción), lo que mejora la transparencia del EIA y el diálogo con comunidades y autoridades, reduciendo la brecha entre la documentación técnica y la percepción de los impactos.

\end{itemize}


\subsection{¿Cómo utilizar BIM para la descripción del proyecto que se presenta en el EIA?}

En un Estudio de Impacto Ambiental, la \textit{Descripción del Proyecto} debe explicar con claridad qué se quiere construir, dónde y cómo. BIM puede ser la fuente única de información desde la cual se generan planos, tablas y figuras, manteniendo coherencia entre texto, dibujos y planillas.

\subsubsection{ Localización y emplazamiento}

El modelo BIM puede georreferenciarse y vincularse a cartografía oficial, ortofotos y modelos digitales de terreno.  
Así se representan en 3D la ubicación del proyecto, sus accesos, límites y fajas de servidumbre, y se visualiza su relación con elementos sensibles del entorno (cursos de agua, áreas protegidas, centros poblados, infraestructura existente). Estos insumos permiten elaborar planos de ubicación y secciones típicas directamente desde el modelo.

\subsubsection{Componentes del proyecto: obras principales y auxiliares}

Las distintas partes del proyecto (edificaciones, estructuras, caminos, tuberías, líneas eléctricas, equipos, instalaciones de faena, etc.) se modelan como categorías o sistemas dentro del BIM.  
A partir de ello se obtienen listados automáticos de obras principales y auxiliares con sus dimensiones relevantes (largo, área, volumen, capacidad, potencia). Los elementos temporales pueden asignarse a fases específicas, diferenciándolos de las obras permanentes. Al modificar un parámetro de diseño, el modelo actualiza de forma consistente planos y tablas usados en la descripción del proyecto.

\subsubsection{Fases: construcción, operación y cierre}

Mediante BIM 4D es posible asociar el cronograma a los elementos del modelo.  
En construcción, las simulaciones muestran la secuencia de actividades (despeje, movimiento de tierras, montajes), ayudando a explicar en qué momentos se generan ciertos impactos.  
En operación, el modelo describe el estado final del proyecto y las capacidades de cada sistema.  
Para el cierre, se pueden representar las obras a desmantelar, las estabilizaciones y las recuperaciones paisajísticas, generando planos específicos que respaldan el plan de abandono.

\subsubsection{Procesos, insumos y flujos}

Elementos del modelo pueden vincularse a datos de consumo de agua, energía, combustibles y materias primas.  
También se asocian parámetros de operación (caudales, emisiones, niveles de ruido, temperaturas) a equipos y sistemas, que luego se usan en la modelación de impactos.  
Con vistas 3D simplificadas se pueden elaborar esquemas de procesos (tratamiento de aguas, ventilación, transporte interno) que sustituyen o complementan los diagramas de bloques tradicionales.

\subsubsection{Representación gráfica y apoyo a la comprensión}

Desde el modelo BIM se generan plantas, cortes y elevaciones coherentes, además de vistas 3D y recorridos virtuales que muestran la volumetría del proyecto y su inserción en el paisaje.  
Las animaciones 4D de la secuencia constructiva son útiles en la participación ciudadana y en reuniones técnicas, al hacer más comprensible el desarrollo de las obras.

\subsection{Uso de BIM en la Presentación de la Línea Base}

En un Estudio de Impacto Ambiental, la \textbf{línea base} describe el estado del medio antes de la ejecución del proyecto. Incluye componentes físicos, bióticos y sociales que puedan verse afectados.  
Su calidad es clave: si la línea base es pobre o poco clara, la evaluación de impactos será débil.

BIM puede apoyar fuertemente esta etapa si se concibe el modelo no sólo como una representación del proyecto, sino también como un contenedor estructurado de información del \emph{entorno}. La idea central es integrar, alrededor del modelo BIM, la cartografía, los levantamientos y los datos medidos en terreno, de manera que la línea base se presente de forma espacialmente coherente y fácil de entender.

\subsubsection{Integración BIM--GIS y representación del territorio}

El primer paso es \textbf{georreferenciar} el modelo BIM y vincularlo con información GIS.  
Sobre el terreno digital (modelo digital de elevación, curvas de nivel, ortofotos) se pueden cargar capas como:

\begin{itemize}
    \item Uso de suelo y coberturas de vegetación,
    \item Hidrografía (ríos, esteros, humedales, zonas inundables),
    \item Geología y riesgos naturales (fallas, inestabilidades de ladera),
    \item Límites administrativos, áreas protegidas y zonas con restricciones normativas.
\end{itemize}

Estas capas se vinculan al modelo BIM como referencias externas o nubes de puntos.  
De este modo, las figuras de la línea base (mapas de ubicación de elementos relevantes) se generan coherentemente con la misma geometría usada en la descripción del proyecto.

\subsubsection{Componentes físicos: clima, aire, agua, suelo y ruido}

Para la \textbf{línea base física}, los puntos de medición y estaciones se pueden representar como objetos BIM (familias de “punto de monitoreo”) con atributos asociados:

\begin{itemize}
    \item Coordenadas y cota.
    \item Tipo de variable medida (calidad de aire, ruido, caudal, calidad de agua, etc.).
    \item Fechas de campaña y valores estadísticos (promedios, máximos, percentiles).
\end{itemize}

Por ejemplo, los \emph{puntos de ruido} se ubican en el modelo alrededor de receptores sensibles (viviendas, escuelas, hospitales). Cada punto almacena los niveles medidos e información de respaldo.  
Las figuras de la línea base (mapas de isófonas, ubicación de estaciones hidrométricas, etc.) pueden generarse combinando vistas 3D/2D del modelo con simbología estándar.

En hidrología, cursos y cuerpos de agua se representan como elementos lineales o volumétricos vinculados a datos de caudal, régimen y calidad. Esto ayuda a entender la relación espacial entre el proyecto y las unidades hidrográficas relevantes.

\subsubsection{Componentes bióticos: flora, fauna y hábitats}

Para la \textbf{línea base biótica}, el entorno se puede dividir en \textbf{unidades de hábitat} modeladas como volúmenes o polígonos sobre el terreno BIM:

\begin{itemize}
    \item Cada unidad contiene atributos como tipo de vegetación, estado de conservación, presencia de especies protegidas o endémicas.
    \item Los puntos o transectos de muestreo de flora y fauna se incorporan como objetos con información de campañas, especies registradas y abundancias.
\end{itemize}

Esto permite visualizar claramente qué hábitats y comunidades biológicas se encuentran dentro del área de influencia y cómo se distribuyen respecto del proyecto.  
Además, facilita cuantificar superficies por tipo de hábitat, información que luego se usa para evaluar impactos y diseñar medidas de mitigación o compensación (por ejemplo, reforestaciones equivalentes).

\subsubsection{Medio humano: asentamientos, usos y sensibilidad social}

En el \textbf{medio humano}, BIM puede representar:

\begin{itemize}
    \item Localización de poblados, viviendas dispersas, equipamientos (escuelas, postas, sedes comunitarias).
    \item Rutas de acceso, caminos vecinales y redes de transporte.
    \item Usos del territorio relevantes: actividades productivas, turismo, recreación, sitios de valor cultural.
\end{itemize}

Estos elementos se pueden modelar como volúmenes simples o bloques referenciados, con atributos que describan población, número de viviendas, principales actividades económicas y nivel de sensibilidad frente a cambios (por ejemplo, comunidades que dependen del agua de un determinado estero).

De esta forma, las figuras de la línea base social (mapas de asentamientos, redes viales, zonas de influencia) se obtienen directamente a partir del modelo, manteniendo la misma base espacial que el resto del EIA.

\subsubsection{Coherencia, trazabilidad y actualización}

Un beneficio importante de usar BIM en la línea base es la \textbf{trazabilidad}.  
Cada mapa, tabla o gráfico puede vincularse al objeto BIM o a la capa de datos de la cual proviene, registrando fuente, fecha y responsable de la información.  
Si en una etapa posterior se actualiza un dato (por ejemplo, una nueva campaña de monitoreo o una corrección de cartografía), el modelo permite identificar rápidamente qué figuras y secciones del EIA se ven afectadas.

Además, el modelo se puede reutilizar en etapas de seguimiento: los mismos puntos de monitoreo definidos en la línea base sirven para controlar el cumplimiento de compromisos ambientales durante la construcción y operación, manteniendo continuidad entre diagnóstico, evaluación y seguimiento.








\newpage
%\input{referencias}



%Aspectos formales (valor 0,5 puntos)
% REFERENCIAS, utilizar solo si es necesario

%\newpage
%\begin{references}
%\end{references}

% ANEXOS, utilizar solo si es necesario

%\begin{anexo}
%\end{anexo}

% FIN DEL DOCUMENTO
\end{document}
